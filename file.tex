\documentclass[12pt,  openany]{book}
\pagestyle{plain}
\usepackage[utf8]{inputenc}
\usepackage[T1]{fontenc}

\usepackage{extsizes}
\usepackage[english,russian]{babel}
\usepackage{geometry}
\usepackage{graphicx}
\usepackage{hyphsubst}
\usepackage{tempora}
\usepackage{amsmath}
\usepackage[nottoc,notlot,notlof,numbib]{tocbibind}
\usepackage{natbib}
\usepackage{indentfirst}
\usepackage{titlesec}
\titleformat{\chapter}[display]
  {\normalfont\bfseries}{}{0pt}{\Huge}
\usepackage{listings}
\usepackage{xcolor}
\usepackage{hyperref}
\hypersetup{
    colorlinks,
    citecolor=black,
    filecolor=black,
    linkcolor=black,
    urlcolor=black
}

\definecolor{codegreen}{rgb}{0,0.6,0}
\definecolor{codegray}{rgb}{0.5,0.5,0.5}
\definecolor{codepurple}{rgb}{0.58,0,0.82}
\definecolor{backcolour}{rgb}{0.98,0.98,0.98}

\lstdefinestyle{mystyle}{
    frame=single,
    aboveskip=5mm,
    belowskip=5mm,
    backgroundcolor=\color{backcolour},   
    commentstyle=\color{codegreen},
    keywordstyle=\color{blue},
    numberstyle=\small\color{codegray},
    stringstyle=\color{codepurple},
    basicstyle=\ttfamily\footnotesize,
    breakatwhitespace=false,         
    breaklines=true,                 
    captionpos=b,                    
    keepspaces=true,                 
    numbers=left,                    
    numbersep=7pt,                  
    showspaces=false,                
    showstringspaces=false,
    showtabs=false,                  
    tabsize=2
}

\lstset{style=mystyle}
  
\geometry{
    a4paper,
    left=30mm,
    top=20mm,
    right=15mm,
    bottom=20mm
}
\linespread{1.5}

\title{Проверка консистентности и изоляции транзакций в распределенных системах}
\author{Исаева Анна}
\date{2021}
\setlength{\parindent}{1.25cm}

\begin{document}
\maketitle
\chapter*{Аннотация}
\par
аннотация \newline
объем аннотации <  1500 символов,  отразить цели и задачи работы, полученные результаты, рекомендации, предложенные на основании данной работы
\setcounter{page}{2}
\tableofcontents
\clearpage



\chapter{Введение}

Существует спрос на инструменты проверки изоляции транзакций, так  как базы данных не обеспечивают тот уровень изоляции, на который претендуют.  Почему? Чем они могут быть полезны?
\begin{itemize}
\item Пользователю(разработчику приложения, хранящего свои данные в базе данных) хочется научиться понимать про ту или иную базу данных, насколько она соответствует документации (это необходимо для того, чтобы выбрать базу, максимально удовлетворяющую потребностям; пользователя)
\item Удобный инструмент для тестирования согласованности может существенно помочь на этапе разработки распределенных систем, возможно, стать одним из этапов CI/CD процесса;
\item Jepsen проводит свои исследования независимо и в соответствии с их этической политикой. Это вносит большой вклад в сообщество, помогает в развитии и совершенствовании распределенных систем.
\end{itemize}
Большинство распределенных систем стремятся к достижению баланса между временем выполнения операций и согласованностью.  Один из инструментов для проверки гарантии согласованности - это инструмент хаос-тестирования Jepsen.  Хаос-тестирование ---  это тестирование путем внесения в систему незапланированных сбоев.  Наблюдая за поведением системы, можно понять, как сделать распределенную систему более надежной. Хаос тестирование это важная часть тестирования, потому что помогает выявить состояния гонки (race condition), которые сложно иначе обнаружить в процессе разработки.

\section{Цели работы}
\begin{itemize}
  \item Получить опыт работы с выбранным инструментом проверки свойств транзакций (Jepsen);
  \item Изучить выполненные данным инструментом исследования различных баз данных;
  \item Исследовать с помощью выбранного инструмента реальную базу данных (Azure Cosmos DB), которая еще не была проанализирована;
  \item Сравнить уровень согласованности, заявленный в документации, и уровень, установленный с помощью тестов.
\end{itemize}

\section{Основные понятия}
Введем основные понятия,  необходимые в дальнейшем.

\emph{Распределенная система} ---  

\emph{Хаос-тестирование} ---  это тестирование путем внесения в распределенную систему незапланированных сбоев.

\emph{Процесс} ---  

\emph{Операция} --- переход из одного состояние в другое. 

\emph{Атомарная операция} --- операция в общей области памяти, которая завершается за один шаг относительно других потоков, имеющих доступ к этой области памяти.  Во время выполнения такой операции над переменной ни один поток не может наблюдать изменение наполовину завершенным. Неатомарные операции не дают такой гарантии.

\emph{Параллелизм} ---  

\emph{Сбой} ---  

\emph{История} ---  

\emph{Модель согласованности} --- набор гарантий о предсказуемости результатов операций чтения,  записи и изменения данных,  позволяющий рассуждать о поведении компьютерной программы.

\section{План работы}
В \textit{Главе 2} будут рассмотрены 
\par
В \textit{Главе 3} будет представлен анализ
 \par
\textit{Глава 4} включит в себя .

\chapter{Синхронизация в базах данных}
\section{Базы данных}
\section{Модели согласованности}
(добавить ссылку на сайт jepsen)
\subsection{Строгая сериализуемость (англ.  \textit{Strict Serializability})}


\subsection{Сериализуемость(англ.  \textit{Serializability})}


\section{Нарушения согласованности}
... какие проблемы бывают с согласованностью

\chapter{Методология проверки согласованности распределенных систем}
\section{Jepsen}
\chapter{Исследование согласованности Azure Cosmos DB}
\section{Azure Cosmos DB}

\section{Уровни согласованности}
Azure Cosmos DB предоставляет 5 уровней согласованности (добавить ссылку на документацию cosmos db).
\subsection{strong}
Данные синхронно реплицируются на все реплики в режиме реального времени. Гарантирует линеаризацию. Это означает, что порядок операций сохраняется, и считывания гарантированно возвращают самую последнюю версию элемента в базе данных. Клиент всегда получает последние изменения в данных по запросу. Никогда не будет видима незафиксированная или частично измененная запись. Самая низкая производительность и доступность.
\subsection{bounded staleness}
Данные реплицируются асинхронно с заданным окном устаревания, определяемым либо количеством записей, либо периодом времени. Запрос на чтение может отставать либо на определенное количество операций записи, либо на заранее определенный период времени. Однако при чтении гарантируется соблюдение последовательности данных. По мере приближения окна устаревания репликация данных запускается в учетной записи базы данных, заставляющей базу данных обновлять новые записи с момента последнего изменения. Низкая доступность из-за задержки синхронизации разных регионов. Более хорошая производительность.
\subsection{session}
Это уровень согласованности по умолчанию. Это обеспечивает сильную согласованность для сеанса приложения с одним и тем же токеном сеанса. Это означает, что все, что написано сеансом, также вернет последнюю версию для чтения из того же сеанса. Доступность данных относительно высока и более низкой задержкой и более высокой пропускной способностью, чем bounded staleness. Данные из других сеансов поступают в правильном порядке, просто не гарантируется, что они будут актуальными.
\subsection{consistent prefix}
Эта модель согласованности аналогична bounded staleness, только без гарантии задержки. Реплики гарантируют согласованность и порядок записи, однако данные не всегда актуальны. Эта модель гарантирует, что пользователь никогда не увидит неупорядоченную запись. Высокая доступность и низкая задержка.
\subsection{eventual}
Для операции чтения нет гарантии порядка данных, а также отсутствует гарантия того, сколько времени может потребоваться для репликации данных.  Это слабая форма согласованности, так как могут быть считаны более старые значения чем те, которые были считаны раньше.
Высокая доступность, самая высокая пропускная способность и низкая задержка.



\chapter{Заключение}
Выводы
%\bibliographystyle{unsrt}
%\bibliography{references}

\chapter{Приложения}


\end{document}
